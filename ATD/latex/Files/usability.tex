%.-------------------------------------------------------------------------------------------------------------------------------------------------------------
This section presents an analysis of the usability of the project. This section should be considered as suggestions for a further release, some ideas and hints to increase the user experience.
\subsection{Mobile Application Testing}
The application is well structured and provides a good environment to test out the functionalities of the system.
Some functions may be changed to simplify testing :
\begin{itemize}
\item Save last connected ip address: when closing the application the ip gets deleted and the default ip is automatically inserted and used to connect to the server. This slows down testing of behaviour of the application for example testing the behaviour of closing it and reopening it. If the mobile phone is set so that application in background are stopped the ip must be reinserted to make the application communicate with the right ip.
\item Make the Camera choosable: when testing the application having a fixed camera can cause unexpected and undesirable problems. On some phones the primary camera is the internal camera. This caused our testing phase of report making really difficult since we had to make clear photos of the cars and license plates. That was quite a difficult task since internal camera's makes photos with less quality and are usually focused on taking a photo of a person not of a car which can be really big and this reduces too the accuracy of the photo making the testing almost impossible since most photos gets rejected.
\end{itemize}
An additional point is that the interface is really good looking at first glance but seems to be tested only on a certain specific type of screens. Because some screens shows some really bad looking overflow messages for email addresses of accounts which were already available in the database provided.
//TODO photo here

\subsection{Mobile Application for Users}
Here we present some ideas and hints to improve user experience for final users and some improvements to the functionalities to make them more clear and usable:
\begin{itemize}
\item Making the License plate input field Auto Capitalizing : The application for making a report requires the user to manually input the license plate of a violation. The input is not accepted if letters are in lower case, but since the input is a normal text input a normal phone keyboard capitalizes the first letter and puts the other letters in lowercare. This can slow down the creation of a report , the input field may set programmatically the capitalization of the letters inside the license plate. Another way to handle this problem could be just modifying to uppercase all the letters in the license plate server side or client side indifferently.
\item Taking a photo with a camera of choice : The possibility of choosing the camera to take a photo on a phone is required to make the process as easy as possibile for all possible users on the most different devices as possible. 
\item Check overflow behaviour of components : A fixed behaviour should be decided for overflowing elements and texts. Making buttons proportional to the screen size and making text size also proportional to the size may be a possible choice, making scrollbars to see overflowing elements is also a possible behaviour to handle overflows.
\item Top Left Alert button : the button which gives the possibility to review a photo and increase reliability of a report should be visible only if there is a real report to be reviewed. Otherwise for a user which doesn't know how it works may see it and click it several time getting no feedback from the application. Another possibile way to avoid this unexplained button may be setting a default screen pop-up telling the user that there are no reports to be reviewed. Explaining this way the meaning of the component in the user interface.
\item Expect the unexpected : On the report map screen there is a map where the violations are shown with their id , their date and type. In this screen it is possible using a menu to choose to filter violations. The filtering used properly works really well. it is possible to delete the starting or ending date of the filtering . If a user deletes one date of the two and presses filter the screen shows a loading icon on bottom left and after several minutes of loading no responses seems to arrive from the server. Filter should be disabled if both dates are not available or a default date should be decided for both fields. For example one week before as default starting date and the current date as default final date. Another unexpected behaviour is that after changing filter options and setting a certain violation type the map reload the violations but if there is another change in the violation type and filter is clicked there is no loading and no changes occur on the map, the changes happens only when the map is moved.
The filter button should be enabled again after a filter option is changed, this functionality may be useful for municipalities, one municipality may want  to know how different violations are located on their territory. Not having the possibility to filter without moving the map can make the experience tedious and less intuitive.
\item Don't make the user learn , give them what they already know : On the report map it is not possible to directly move in direction north-south because vertical scrolling moves a menu on the top of the map. This behaviour is really unexpected by the user and user must learn that to move north-south they must first move a little east-west and keep moving on the screen vertically to actually move in the desired direction. Final users can be really difficult to please and making them learn new behaviours which are not compatible with behaviours they already are familiars with can really make the difference from a used and a not used application. To fix this issue a button may be placed on top of the screen to make the menu appear or as an alternative an area which is easily recognizable may be put on top and dragging this area makes the menu appear and disappear. This way the map itself remains untouched while the experience of the user increases and a more natural interaction with the map is possible.
\end{itemize}
\subsection{Interfaces}
The design of the interfaces in the application is quite clear and there are no distractions from the actual activity of the user which makes the experience of the user linear and creates less space for unexpected behaviours. 
\newline
There is unity between the various components in the screens so every interface has its compoents arranged in a clear way and makes the interactions clear to the user.
\newline
Functionalities are showed on the main screen with a scaling hierarchy ,showing in big the main functionalities which are making reports and accessing the map with the reports and givin less importance to functionalities like reviewing a report and modifying account.
\newline
All the interfaces are have its components placed in such a way to preserve a balance in the screens which improves the user experience.

