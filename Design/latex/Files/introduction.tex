
%---------------------------------------------------
\subsection{Purpose}
The purpose of this document is going more in the technical details than the RASD concerning SafeStreets application.
\newline
The Design Document gives more details about the design giving guidelines over the overall architecture of the system.
This document  aims to identify the core design choices for developing the system:
\begin{itemize}
\item The high level architecture
\item The components and their interfaces
\item The design patterns
\item The Interaction between the components
\item Planning for implementation, integration and testing of the system
\end{itemize}
A mapping of the requirements to the architecture's components is also given in the underneath chapters. 

%---------------------------------------------------
\subsection{Scope}
\subsubsection{Description of the given problem}
SafeStreets is a crowd-sourced application whose intention is to notify the authorities when traffic
violations occur. Citizens, thanks to the system, will be able to send information about violations to the
authorities who will take actions against them. In this way, the service provided by the authorities can be
improved because they will receive notifications through the app. The sources of notifications are the
Citizens who take photos of violations and send them to the authorities through
the application. The information provided by users are integrated with other suitable information and are
stored by the service. The system also runs an algorithm to read the license plate of the vehicle in the
photos. All collected data can be seen by Citizens and authorities to find which streets are the safest.
Users can have different levels of visibility: authorities must be able to know the license plates of vehicles
in the photos, while normal users can only see data in the form of statistics. Moreover, data are sent to the municipal district so that important information can be extracted, and the system makes statistics and suggestion in order to make decisions to improve the safety of the area. Finally, the system will have to be easy to use, reliable
and highly scalable to fit perfectly with the mutable context in which it will be used.


%---------------------------------------------------
\subsection{Definitions, Acronyms, Abbreviations}
%-----------------------------------------------------------------
\subsubsection {Definitions}
\begin{itemize}
\item	Violation: parking violations which can be notified by Citizens to authorities
\item	Report: Notification sent by Citizens to the system
\item	Mapping System: external software that provides maps and directions to reach the position of a violation
\item	Licence plate Recognition Algorithm: calculation process that identifies the alphanumeric number on license plate
\item	Spam: a series of messages that are undesired
\item	App: application software 
\item	Blocked: means that the account is banned for a given period
\item	Metadata: data about a violation. Position , date,  time and the username of Citizen. 
\item	Assignment : Work Request for authorities generated upon the receiving of a notification made by Citizens.
\end{itemize}
%---------------------------------------------------
	\subsubsection {Acronyms}
\begin{itemize}
\item	RASD: Requirement Analysis and Specification Document.
\item      DD: Design Document
\item	API: Application Programming Interface
\item	GPS: Global positioning system
\item	HTTP: HyperText Transfer Protocol
\item	HTTPS: HyperText Transfer Protocol over Secure Socket Layers
\item	UML: Unified Modeling Language
\item 	JSON: JavaScript Object Notation
\item 	UI: User Iterface
\item 	SQL:Structured Query Language

\end{itemize}
%---------------------------------------------------
	\subsubsection {Abbreviations}
\begin{itemize}
\item	[Rn]: n-th requirement.
\end{itemize}
%---------------------------------------------------
\subsection {Revision History}
\begin{itemize}
\item	DDv1.0 delivered on 9/12/2019
\end{itemize}
%---------------------------------------------------
\subsection {Reference Documents}
\begin{itemize}
\item 	RASD version 1
\item	Specification Document: “Assignments AA 2019-2020.pdf”.
\item	IEEE Standard for Information Technology—Systems Design—Software Design Descriptions
\end{itemize}
%---------------------------------------------------
\subsection{Document Structure}
This chapter debates about contents and structure of DD, indeed this document is divided in seven different sections:
\begin{enumerate}
\item	The Introduction provides a general appearance of the systems defining which are the goals to reach.
\item Architectural Design: This chapter illustrates the main components of the system and the relationships between them, supplying information about their workflow and deployment. This part of the document also focuses on the main architectural styles and patterns.
\item User Interface Design: This chapter provides a general idea of how the user interfaces will be structured.
\item  Requirements Traceability:This chapter explains how the requirements defined in the RASD are correlated to the design elements that are defined in this document.
\item  Implementation, integration and test plan: This chapter identifies the order of implementation and integration of the various components of the system. The testing of those components is also described in this chapter.
\item  Effort Spent : This chapter shows the amount of hours spent by each member of the group to write the document
\item  Appendix : In this chapter the tools used to create the documentation are listed.
\end{enumerate}
